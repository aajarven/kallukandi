% --- Template for thesis / report with tktltiki2 class ---
% 
% last updated 2013/02/15 for tkltiki2 v1.02

\documentclass[12pt,finnish]{tktltiki2}

% tktltiki2 automatically loads babel, so you can simply
% give the language parameter (e.g. finnish, swedish, english, british) as
% a parameter for the class: \documentclass[finnish]{tktltiki2}.
% The information on title and abstract is generated automatically depending on
% the language, see below if you need to change any of these manually.
% 
% Class options:
% - grading                 -- Print labels for grading information on the front page.
% - disablelastpagecounter  -- Disables the automatic generation of page number information
%                              in the abstract. See also \numberofpagesinformation{} command below.
%
% The class also respects the following options of article class:
%   10pt, 11pt, 12pt, final, draft, oneside, twoside,
%   openright, openany, onecolumn, twocolumn, leqno, fleqn
%
% The default font size is 11pt. The paper size used is A4, other sizes are not supported.
%
% rubber: module pdftex

% --- General packages ---

\usepackage[utf8]{inputenc}
\usepackage[T1]{fontenc}
\usepackage{lmodern}
\usepackage{microtype}
\usepackage{amsfonts,amsmath,amssymb,amsthm,booktabs,color,enumitem,graphicx}
\usepackage[pdftex,hidelinks]{hyperref}

% Automatically set the PDF metadata fields
\makeatletter
\AtBeginDocument{\hypersetup{pdftitle = {\@title}, pdfauthor = {\@author}}}
\makeatother

% --- Language-related settings ---
%
% these should be modified according to your language

% babelbib for non-english bibliography using bibtex
\usepackage[fixlanguage]{babelbib}
\selectbiblanguage{finnish}

% add bibliography to the table of contents
\usepackage[nottoc]{tocbibind}
% tocbibind renames the bibliography, use the following to change it back
\settocbibname{Lähteet}

% spacing
\usepackage{setspace}
\onehalfspacing

% enumitem
\usepackage{enumitem}

% algorithm snippets
\usepackage{algorithm}
\usepackage{algpseudocode}
\usepackage{caption}
\floatname{algorithm}{Algoritmi}

% copy paste from http://tex.stackexchange.com/questions/33866/algorithm-tag-and-page-break
\DeclareCaptionFormat{algor}{%
  \hrulefill\par\offinterlineskip\vskip1pt%
    \textbf{#1#2}#3\offinterlineskip\hrulefill}
\DeclareCaptionStyle{algori}{singlelinecheck=off,format=algor,labelsep=space}
\captionsetup[algorithm]{style=algori}

% --- Theorem environment definitions ---

\newtheorem{lau}{Lause}
\newtheorem{lem}[lau]{Lemma}
\newtheorem{kor}[lau]{Korollaari}

\theoremstyle{definition}
\newtheorem{maar}[lau]{Määritelmä}
\newtheorem{ong}{Ongelma}
\newtheorem{alg}[lau]{Algoritmi}
\newtheorem{esim}[lau]{Esimerkki}

\theoremstyle{remark}
\newtheorem*{huom}{Huomautus}


% --- tktltiki2 options ---
%
% The following commands define the information used to generate title and
% abstract pages. The following entries should be always specified:

\title{Monte Carlo -puuhaku ja pelitekoälyt}
\author{Kalle Viiri}
\date{\today}
\level{Aine}
\abstract{}

% The following can be used to specify keywords and classification of the paper:

\keywords{Monte Carlo, puuhaku, tekoäly}

% classification according to ACM Computing Classification System (http://www.acm.org/about/class/)
% This is probably mostly relevant for computer scientists
% uncomment the following; contents of \classification will be printed under the abstract with a title
% "ACM Computing Classification System (CCS):"
% \classification{}

% If the automatic page number counting is not working as desired in your case,
% uncomment the following to manually set the number of pages displayed in the abstract page:
%
% \numberofpagesinformation{16 sivua + 10 sivua liitteissä}
%
% If you are not a computer scientist, you will want to uncomment the following by hand and specify
% your department, faculty and subject by hand:
%
% \faculty{Matemaattis-luonnontieteellinen}
% \department{Tietojenkäsittelytieteen laitos}
% \subject{Tietojenkäsittelytiede}
%
% If you are not from the University of Helsinki, then you will most likely want to set these also:
%
% \university{Helsingin Yliopisto}
% \universitylong{HELSINGIN YLIOPISTO --- HELSINGFORS UNIVERSITET --- UNIVERSITY OF HELSINKI} % displayed on the top of the abstract page
% \city{Helsinki}
%


\begin{document}
% --- Front matter ---

\frontmatter      % roman page numbering for front matter

\maketitle        % title page
\makeabstract     % abstract page

\tableofcontents  % table of contents

% --- Main matter ---

\mainmatter       % clear page, start arabic page numbering

\section{Johdanto}

Pelillä tarkoitetaan peliteoriassa ongelmaa, jossa yksi tai useampi toimija vaikuttavat päätöksillään lopputulokseen, jolla on jokin määritelty arvo kullekin pelaajista~\cite{browne}. Mahdollisten pelitilanteiden joukkoa merkataan $S$:llä. Pelillä on alkutila $s_0 \in S$ ja joukko lopputiloja $S_T \subseteq S$. Mahdollisten siirtojen joukkoa merkitään $A$:lla, ja tilasiirtymäfunktio $f : S \times A \rightarrow S$ määrittää, mihin tilaan päädytään lähtötilasta milläkin siirrolla.

Pelin lopputuloksen edullisuus pelaajalle määritellään palkintofunktiolla $R : S \rightarrow \mathbb{R}$. Tässä tekstissä käytetään usein käytettyä palkintofunktiota, joka on arvoltaan $0$ pelitilanteissa jotka eivät ole lopputilanteita, ja lopputilanteille $1$ voittavista tilanteista, $0$ tasapeleistä ja $-1$ häviöistä.

Kahden pelaajan peli on kombinatorinen, kun se täyttää seuraavat ehdot~\cite{browne}:

\begin{enumerate}[label=\roman*.]
	\item Nollasummaisuus, eli jokaisessa tilassa pelaajien palkintofunktioiden summa on nolla. Tämä tarkoittaa, että yhden pelaajan parantaessa tulostaan toisen pelaajan tulos heikkenee.
	 \item Täydellinen informaatio, eli tarkka pelitilanne on jatkuvasti kaikkien pelaajien tiedossa.
	 \item Determinismi, eli peliin ei sisälly satunnaistekijöitä.
	 \item Vuorottainen, eli pelaaja voi toiminnallaan vaikuttaa pelitilanteeseen vain määrätyillä vuoroillaan.
	 \item Diskreetti, eli peliä pelataan erillisissä vuoroissa reaaliaikaisuuden sijaan.
\end{enumerate}

Esimerkkejä kombinatorisista peleistä ovat shakki ja go. Kombinatoriset pelit ovat tekoälylle erinomainen tutkimus- ja soveltamisala, sillä niissä on usein yksinkertaiset säännöt, joiden päälle rakentuu kuitenkin edistynyttä strategiaa~\cite{browne}.

Kombinatorinen peli on helppo mallintaa pelipuuna. Pelipuun juurisolmu kuvaa pelin alkutilannetta, ja jokaisen solmun lapset pelitilanteita jotka solmun kuvaamasta pelitilanteesta voi saavuttaa. Pelipuun lehtisolmut ovat lopputilanteita, joissa pelin lopputulos määräytyy. Pelipuuna mallinnettua peliä voi käsitellä tavallisilla puuhakualgoritmeilla~\cite{aima}.

\subsection{Minimax ja heuristiikka}

Perinteinen tapa toteuttaa pelitekoälyjä on \textit{minimax}-haku, jossa tekoäly olettaa kaikkien pelaajien pyrkivän maksimoimaan oman voittonsa toisten pelaajien pelatessa optimaalisesti~\cite{aima}. Tällainen tekoäly valitsisi siis mielummin varmasti tasapeliin johtavan siirron kuin sellaisen, jolla voi vastustajan valinnoista riippuen joko voittaa tai hävitä.

\textit{Minimax}-haun ongelmana on se, että vain puun juurisolmujen arvo on tiedossa pelaajille~\cite{aima}. Yksinkertaisissa peleissä, kuten 3x3-ristinollassa tämä ei ole vielä ongelma, mutta vaikkapa shakin pelipuu haarautuu aivan liian runsaasti jotta hakua voitaisiin jatkaa juurisolmuihin asti. Jotta seuraava siirto saataisiin valittua järkevässä ajassa, on haku katkaistava ennen pelin lopputilanteisiin pääsemistä. Koska haku ei välttämättä pääse pelin lopputilanteisiin saakka, tarvitsee se keinon arvioida keskeneräisiä pelitilanteita lopputilanteiden tavoin.

Keskeneräisen pelitilanteen perusteella voidaan laskea arvio sen edullisuudesta pelaajalle. Tällaisen arvion muodostavaa funktiota kutsutaan heuristiikaksi. Yksinkertainen heuristiikka shakkipeliin voi olla esimerkiksi nappuloiden lukumäärän laskeminen. Tätä heuristiikkaa käyttävä tekoäly osaisi valita tilanteita, joissa se saa materiaalietumatkaa, mutta edistyneemmät heuristiikat osaisivat painottaa tilanteita myös nappuloiden arvon ja sijainnin mukaan. Heuristiikkojen ongelmana on, että hyvän heuristiikan muodostaminen vaatii kehittäjältään paljon ymmärrystä pelistä. Joihinkin peleihin, esimerkiksi gohon, ei tunneta heuristiikkoja joilla tekoäly olisi kilpailukykyinen vahvoja ihmispelaajia vastaan~\cite{browne}.


\section{Monte Carlo -puuhaku}

Monte Carlo -puuhaku~\cite{browne} (\textit{Monte Carlo Tree Search}, jatkossa MCTS) on puuhakumenetelmä, jossa heuristiikan sijaan pelitilanteita arvioidaan simuloimalla niistä satunnaisotannalla pelejä lopputilanteeseen asti. Perusajatuksena on, että toistamalla simulaatioita ja tarkastelemalla lopputuloksia saadaan toimiva arvio pelitilanteen arvosta pelaajalle.

MCTS-menetelmässä pelipuuta rakennetaan iteratiivisesti tarpeen mukaan. Haku aloitetaan pelipuulla, jossa on pelkästään pelin alkutilaa $s_0$ vastaava solmu $v_0$. Puuta laajennetaan etsimällä juuresta alkaen pelitilanteita, joita vastaavilla solmuilla on pelipuuhun liittymättömiä lapsia. Laajennettavan solmun valinta ei väistämättä tapahdu tasapuolisesti, vaan esimerkiksi haaroja, joista on saatu lupaavia tuloksia, voidaan laajentaa muita enemmän.

Kun laajennettava solmu on valittu, puuhun liitetään sen uudeksi lapseksi jokin siitä seuraava pelitilanne. Tästä lapsesta suoritetaan pelin loppuun päättyvä simulaatio, jossa molempien pelaajien vuorot pelataan satunnaisilla siirroilla~\cite{browne}. Pelin lopettavaan tilanteeseen päädyttäessä simulaatio voi tarkistaa tunnetuista voittoehdoista tilanteesta seuraavan lopputuloksen $\Delta$. Saatu $\Delta$ palautetaan pelipuussa vastavirtaan juureen asti. Matkalla olevien solmujen odotusarvoja päivitetään uuden $\Delta$-arvon mukaan, jolloin solmuille muodostuu simulaatioiden jatkuessa tarkempi arvio niiden kuvaamien pelitilanteiden kannattavuudesta pelin voittamisen kannalta~\cite{browne}.


Algoritmissa \ref{MCTS} esittellään Brownen ym. yleistetty MCTS~\cite{browne}. Aliohjelmien \textsc{ValitseLapsi}, \textsc{Simuloi}, \textsc{Vastavirta} ja \textsc{ParasLapsi} toiminta on jätetty kuvaamatta, sillä ne vaihtelevat erilaisten MCTS-toteutusten välillä. Esimerkkitoteutus näille esitellään algoritmissa \ref{UCT}.
\newpage
\begin{samepage}
\begin{center}
\captionof{algorithm}{Yleistetty MCTS}
\label{MCTS}
\begin{algorithmic}[0]
\Function{MctsHaku}{$s_0$}
	\State luo pelipuun juuri $v_0$ pelitilanteesta $s_0$
	\While{laskenta-aikaa on jäljellä}
		\State $v_1 \leftarrow$ \Call{ValitseLapsi}{$v_0$}
		\State $\Delta \leftarrow$ \Call{Simuloi}{$s(v_1)$}
		\State \Call{Vastavirta}{$v_1$, $\Delta$}
	\EndWhile
	\State \Return siirto(\Call{ParasLapsi}{$v_0$})
\EndFunction
\end{algorithmic}
\end{center}
\end{samepage}

Haku jatkuu niin kauan kuin laskenta-aikaa riittää tai kunnes suoritus katkaistaan ulkopuolelta. Haun päätyttyä valitaan siirto, joka katsotaan parhaaksi haun aloitustilanteesta suoritettujen simulaatioiden perusteella. Haun päättyessä suoritettavan siirron valintaan mahdollisista siirroista on useita malleja~\cite{browne}. Voidaan esimerkiksi valita siirto, josta seuraavan tilan odotusarvo on korkein, tai siirto, josta on suoritettu eniten simulointeja ja jonka odotusarvo on siksi todennäköisesti lähimpänä sen todellista arvoa.

\section{UCT-menetelmä}

\subsection{Monikätisen rosvon ongelma}

Monikätisen rosvon optimointiongelmassa~\cite{browne} (\textit{multi-armed bandit problem}) pelaajalla on peliautomaatti, jossa on $K$ vipua. Jokaiselle vivulle on määritelty pelaajalle tuntematon todennäköisyysjakauma, jonka mukaan pelaaja saa palkinnon vetäessään vipua. Ongelmana on löytää strategia, jolla minimoidaan pelkästään parasta vipua pelaavan strategian kokonaispalkinnon etumatka omaamme nähden.

Tehokkaan pelistrategian löytäminen monikätisen rosvon ongelmaan edellyttää algoritmia, joka osaa pelata paljon vivuilla joista on jo saatu hyviä tuloksia. Samalla kuitenkin sen on tutkittava välillä muitakin vipuja, jotta mahdollisesti huomaamatta jääneet vivut voidaan hyödyntää. Tällaisen strategian esittelevät Auer ym. joiden luottamusylärajan menetelmä~\cite{auer} (\textit{Upper Confidence Bounds}, jatkossa UCB1) valitsee pelattavaksi vivun joka maksimoi lausekkeen
\begin{equation}
\text{UCB1} = \overline{X}_j + \sqrt{\frac{2 \ln n}{n_j}}
\end{equation}
arvon, missä $X_j$ on vivun $j$ tuottamien palkintojen keskiarvo, $n_j$ on vivun $j$ pelikertojen määrä tähän mennessä ja $n$ on koko pelikoneen pelikertojen määrä. Tasapelit ratkaistaan toteutuksissa tyypillisesti satunnaisesti, ja kun $n_j = 0$ katsotaan että $\text{UCB1} = \infty$ eli kaikkia vipuja kokeillaan vähintään kerran ennen kuin yhtäkään pelataan toista kertaa~\cite{browne}.

Lai ja Robbins osoittivat, että useille palkintojakaumille ei ole olemassa monikätisen rosvon pelistrategiaa, jossa ainoastaan parasta vipua pelaavan strategian etumatka kasvaa hitaammin kuin $O(ln)$~\cite{lai, browne}. UCB1:n katumus kasvaa logaritmisesti, joten sitä voi pitää hyvänä ratkaisuna monikätisen rosvon ongelmaan. Se on myös yksinkertainen ja tehokas~\cite{browne}, joten se on mielekäs ratkaisu sovellettavaksi toisiin ongelmiin, kuten puun laajentamiseen MCTS:ssä.

\subsection{Luottamusylärajat puuhaussa}

Kocsisin ym. esittelemä puuhakujen luottamusyläraja -menetelmä (\textit{Upper Confidence Bounds for Trees}~\cite{browne, kocsis} jatkossa UCT) on yleisimpiä algoritmeja MCTS:n toteutukseen. UCT mallintaa pelipuuta laajentaessa solmun valinnan monikätisen rosvon ongelmana, ja käyttää pelistrategiana muunnelmaa UCB1:stä. Tällöin $X_j$ on solmun $j$ kautta suoritettujen simulaatioiden lopputulosten keskiarvo, $n_j$ solmun $j$ läpi suoritettujen simulaatioiden lukumäärä ja $n$ kaiken kaikkiaan suoritettujen simulaatioiden lukumäärä.

UCB1:sta poiketen algoritmille annetaan tutkimusvakio (\textit{exploration term}) $C_p$ joka vaikuttaa algoritmin painotukseen uusien vaihtoehtojen tutkimuksen ja hyväksi tunnettujen hyödyntämisen välillä. Edetessään pelipuussa UCT pyrkii maksimoimaan seuraavan lausekkeen arvon:
\begin{equation}
\text{UCT} = \overline{X}_j + C_p \sqrt{\frac{2 \ln n}{n_j}}
\end{equation}

Browne ym. esittelee UCT-haun algoritmin \ref{UCT} mukaisesti~\cite{browne}. Algoritmissa $s(v)$ on solmun $v$ kuvaama pelitilanne, $f(s(v), a)$ on tilanteesta $s(v)$ siirrolla $a$ saavutettava pelitilanne, $Q(v)$ on $v$:n kautta saavutetun palkinnon kokonaisarvo, $N(v)$ on $v$:n kautta kulkeneiden simulaatioiden lukumäärä ja $\Delta(v, p)$ on pelaajan $p$ palkintoarvo solmussa $v$. Algoritmi palauttaa siirron, joka johtaa lapseen jolla on paras odotusarvo.

\begin{center}
\captionof{algorithm}{UCT-algoritmi}
\label{UCT}
\begin{algorithmic}[0]
\begin{samepage}
\Function{UctHaku}{$s_0$}
	\State luo pelipuun juuri $v_0$ pelitilanteesta $s_0$
	\While{laskenta-aikaa on jäljellä}
		\State $v_1 \leftarrow$ \Call{ValitseLapsi}{$v_0$}
		\State $\Delta \leftarrow$ \Call{Simuloi}{$s(v_1)$}
		\State \Call{Vastavirta}{$v_1$, $\Delta$}
	\EndWhile
	\State \Return siirto(\Call{ParasLapsi}{$v_0, 0$})
\EndFunction
\end{samepage}
\\
\begin{samepage}
\Function{ValitseLapsi}{$v$}
	\While{$v$ ei ole lehti}
		\If{$v$:llä on puuhun lisäämättömiä lapsia}	
			\State \Return \Call{Laajenna}{$v$}
		\Else
			\State $v \leftarrow$ \Call{ParasLapsi}{$v, C_p$}
		\EndIf
	\EndWhile
	\State \Return{$v$}
\EndFunction
\end{samepage}
\\
\begin{samepage}
\Function{Laajenna}{$v$}
	\State valitse $a$ joka on tutkimaton siirto tilasta $s(v)$
	\State lisää $v$:lle lapsi $v'$ siten että $s(v') = f(s(v), a)$
	\State \Return $v'$
\EndFunction
\end{samepage}
\\
\begin{samepage}
\Function{ParasLapsi}{$v, c$}
	\State \Return $v$:n lapsi jolla on suurin $\frac{Q(v')}{N(v')} + c\sqrt{\frac{2 \ln N(v)}{N(v')}}$
\EndFunction
\end{samepage}
\\
\begin{samepage}
\Function{Simuloi}{$s$}
	\While{$s$ ei ole pelin lopputilanne}
		\State valitse $a \in A(s)$ satunnaisesti tasajakaumalla
		\State $s \leftarrow f(s, a)$
	\EndWhile
	\State \Return $R(s)$ 
\EndFunction
\end{samepage}
\\
\begin{samepage}
\Function{Vastavirta}{$v, \Delta$}
	\While{$v$ ei ole \textit{null}}
		\State $N(v) \leftarrow N(v) + 1$
		\State $Q(v) \leftarrow Q(v) + \Delta(v, p)$
		\State $v \leftarrow v$:n vanhempi
	\EndWhile
\EndFunction
\end{samepage}
\end{algorithmic}
\end{center}



\section{MCTS ei-kombinatorisissa peleissä}

\subsection{Epätäydellinen informaatio ja satunnaisuus}

Epätäydellisen informaation peli~\cite{browne} on peli, jossa kaikki pelaajat eivät tiedä kaikkea vallitsevasta pelitilanteesta. Esimerkiksi pokerissa kukin pelaaja näkee vain oman kätensä kortit. Monessa pelissä pelitilanteeseen vaikuttavat myös satunnaistekijät, kuten nopanheitot. MCTS on erilaisten apukeinojen avulla sovellettavissa myös peleihin, joissa informaatio on epätäydellistä ja osa pelitilanteesta määräytyy satunnaistekijöiden kautta~\cite{browne}.

Determinisaatiolla~\cite{browne} tarkoitetaan mahdollisesti epätäydellistä informaatiota ja satunnaistekijöitä sisältävän pelin muuntamista täydellisen informaation deterministiseksi peliksi, jolloin sitä voi käsitellä kombinatorisen pelin tapaan. Determinisoitu peli avaa kaiken informaation kaikille pelaajille ja kiinnittää satunnaistekijöiden arvot tunnetuiksi. Esimerkiksi determinisoidussa pokerissa pelataan avoimin kortein. Menetelmän tarkoituksena on luoda monta determinisoitua pelitilannetta pelistä, soveltaa niihin kombinatorisen pelin MCTS-menetelmää ja käyttää tuloksia toimintamallin valitsemiseen alkuperäisessä pelissä.

Vastustajan strategian mallintaminen on erityisen tärkeää, kun pelin informaatio on epätäydellistä~\cite{browne}. Ainoastaan vastustajan tiedossa oleva informaatio vaikuttaa usein tämän tekemiin siirtoihin, jolloin vastustajan strategiasta voi muodostaa johtopäätöksiä vain tämän tiedossa olevasta informaatiosta. Esimerkiksi Ponsen ym.~\cite{ponsen} käytti pokerissa vastustajan mallinnusta Bayes-verkoilla päätelläkseen vastustajan kädessä olevien korttien todennäköisyysjakauman ja hyödynsi tuloksia determinisoitujen pelien luomisessa MCTS:ä varten.

\subsection{Videopelit}

Reaaliaikaisessa pelissä~\cite{browne} pelitilanne voi edetä, vaikka pelaaja ei toimisi mitenkään. Tämä pakottaa pelaajan toimimaan erityisen nopeasti, sillä liian pitkään mietitty siirto voi muuttua harkinnan aikana huonoksi. Useimmat reaaliaikaiset pelit ovat videopelejä, joissa esiintyy lisäksi muitakin haasteita kuten satunnaisuutta, informaation epätäydellisyyttä ja erityisen suuria määriä mahdollisia pelitilanteita.

Samothrakis ym. tutkivat MCTS-pohjaisen menetelmän toimintaa \textit{Ms. Pac-Man} -pelissä~\cite{samothrakis}. Pelissä pelaaja kuljetta hahmoa sokkelossa tarkoituksena kerätä kartalla olevat pisteet ja vältellä vihollisia. Vastapelaajina on neljä vihollista, joiden osittain satunnainen tekoäly pyrkii nappaamaan pelihahmon. Pelissä ei ole suoranaisia lopputiloja paitsi pelaajan hävitessä, joten MCTS:n simulaatiovaiheen täytyy rajata haku muilla tavoin. Samothrakis ym. ratkaisi ongelman rajaamalla simulaation maksimisyvyyden ja käyttämällä haun katkaisusolmun arviointiin yksinkertaista heuristiikkaa.

Toinen haaste \textit{Ms. Pac-Man} -pelissä on ajoituksen tarkkuus. Samothrakis ym. toteaa, että peli vaatii siirtojen tapahtuvan jopa 50-60 millisekunnin tarkkuudella, mikä asettaa tiukan rajan pelipuussa suoritettavan simulaation määrälle. Tulokset havaittiin kuitenkin erittäin hyviksi, ja MCTS-menetelmällä toimiva tekoäly keräsi testeissä evolutiivisia ja geneetisiä algoritmeja sekä vahvistusoppimista hyödyntäviä menetelmiä kahta kertaluokkaa paremmat pisteet.

\section{Erityisiä hyötyjä}

MCTS:n vahvuus on sen helppo sovellettavuus erilaisiin ongelmiin~\cite{browne}. Koska se ei tarvitse heuristiikkaa, tekoälyn ohjelmoijan ei tarvitse tietää pohjalla olevasta pelistä muuta kuin perussäännöt. Heuristiikan liittäminen MCTS:ään on kuitenkin mahdollista sekä lopputilojen korvaamiseen kuten \textit{Ms. Pac-Man} -esimerkissä että haun tehokkaampaan kohdistamiseen pelipuussa~\cite{browne}.

Useat MCTS-toteutukset, kuten UCT, kohdistavat haun tehokkaasti ja laajentavat pelipuuta eniten suuntiin, joista on saatu lupaavimpia tuloksia~\cite{browne}. Näin algoritmi ei käytä paljoa aikaa huonoja tuloksia antaneiden vaihtoehtojen tutkimiseen. UCT takaa kuitenkin, että kukin solmu valitaan laajennettavaksi nollaa suuremmalla todennäköisyydellä. Tästä johtuen on erittäin epätodennäköistä että yksittäiset huonot kokemukset estävät algoritmia löytämästä todellisuudessa hyviä strategioita.

MCTS ylläpitää arviota siirtojen kannattavuudesta reaaliaikaisesti. Haun voi siis keskeyttää koska tahansa ja saada silti kelvollisen ratkaisun. Pidemmän laskenta-ajan antaminen MCTS:lle parantaa kuitenkin tuloksen laatua, sillä algoritmilla on enemmän aikaa suorittaa simulaatioita ja parantaa niiden avulla arvioitaan siirtojen kannattavuudesta~\cite{browne}.


% --- References ---
%
% bibtex is used to generate the bibliography. The babplain style
% will generate numeric references (e.g. [1]) appropriate for theoretical
% computer science. If you need alphanumeric references (e.g [Tur90]), use
%
% \bibliographystyle{babalpha-lf}
%
% instead.

\bibliographystyle{babplain-lf}
\bibliography{lahteet}


% --- Appendices ---

% uncomment the following

% \newpage
% \appendix
% 
% \section{Esimerkkiliite}

\end{document}

