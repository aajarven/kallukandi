% --- Template for thesis / report with tktltiki2 class ---
% 
% last updated 2013/02/15 for tkltiki2 v1.02

\documentclass[12pt,finnish]{tktltiki2}

% tktltiki2 automatically loads babel, so you can simply
% give the language parameter (e.g. finnish, swedish, english, british) as
% a parameter for the class: \documentclass[finnish]{tktltiki2}.
% The information on title and abstract is generated automatically depending on
% the language, see below if you need to change any of these manually.
% 
% Class options:
% - grading                 -- Print labels for grading information on the front page.
% - disablelastpagecounter  -- Disables the automatic generation of page number information
%                              in the abstract. See also \numberofpagesinformation{} command below.
%
% The class also respects the following options of article class:
%   10pt, 11pt, 12pt, final, draft, oneside, twoside,
%   openright, openany, onecolumn, twocolumn, leqno, fleqn
%
% The default font size is 11pt. The paper size used is A4, other sizes are not supported.
%
% rubber: module pdftex

% --- General packages ---

\usepackage[utf8]{inputenc}
\usepackage[T1]{fontenc}
\usepackage{lmodern}
\usepackage{microtype}
\usepackage{amsfonts,amsmath,amssymb,amsthm,booktabs,color,enumitem,graphicx}
\usepackage[pdftex,hidelinks]{hyperref}

% Automatically set the PDF metadata fields
\makeatletter
\AtBeginDocument{\hypersetup{pdftitle = {\@title}, pdfauthor = {\@author}}}
\makeatother

% --- Language-related settings ---
%
% these should be modified according to your language

% babelbib for non-english bibliography using bibtex
\usepackage[fixlanguage]{babelbib}
\selectbiblanguage{finnish}

% add bibliography to the table of contents
\usepackage[nottoc]{tocbibind}
% tocbibind renames the bibliography, use the following to change it back
\settocbibname{Lähteet}

% spacing
\usepackage{setspace}
\onehalfspacing

% --- Theorem environment definitions ---

\newtheorem{lau}{Lause}
\newtheorem{lem}[lau]{Lemma}
\newtheorem{kor}[lau]{Korollaari}

\theoremstyle{definition}
\newtheorem{maar}[lau]{Määritelmä}
\newtheorem{ong}{Ongelma}
\newtheorem{alg}[lau]{Algoritmi}
\newtheorem{esim}[lau]{Esimerkki}

\theoremstyle{remark}
\newtheorem*{huom}{Huomautus}


% --- tktltiki2 options ---
%
% The following commands define the information used to generate title and
% abstract pages. The following entries should be always specified:

\title{Monte Carlo -menetelmä ja pelitekoälyt}
\author{Kalle Viiri}
\date{\today}
\level{Referaatti}
\abstract{Tiivistelmä.}

% The following can be used to specify keywords and classification of the paper:

\keywords{Monte Carlo, puuhaku, tekoäly}

% classification according to ACM Computing Classification System (http://www.acm.org/about/class/)
% This is probably mostly relevant for computer scientists
% uncomment the following; contents of \classification will be printed under the abstract with a title
% "ACM Computing Classification System (CCS):"
% \classification{}

% If the automatic page number counting is not working as desired in your case,
% uncomment the following to manually set the number of pages displayed in the abstract page:
%
% \numberofpagesinformation{16 sivua + 10 sivua liitteissä}
%
% If you are not a computer scientist, you will want to uncomment the following by hand and specify
% your department, faculty and subject by hand:
%
% \faculty{Matemaattis-luonnontieteellinen}
% \department{Tietojenkäsittelytieteen laitos}
% \subject{Tietojenkäsittelytiede}
%
% If you are not from the University of Helsinki, then you will most likely want to set these also:
%
% \university{Helsingin Yliopisto}
% \universitylong{HELSINGIN YLIOPISTO --- HELSINGFORS UNIVERSITET --- UNIVERSITY OF HELSINKI} % displayed on the top of the abstract page
% \city{Helsinki}
%


\begin{document}

% --- Front matter ---

\frontmatter      % roman page numbering for front matter

\maketitle        % title page
\makeabstract     % abstract page

\tableofcontents  % table of contents

% --- Main matter ---

\mainmatter       % clear page, start arabic page numbering

\section{Peliteoria ja pelipuu}

Pelillä tarkoitetaan peliteoriassa ongelmaa, jossa yksi tai useampi toimija vaikuttavat päätöksillään lopputulokseen, jolla on jokin määritelty arvo kullekin pelaajista. Peli on kombinatorinen, kun se täyttää seuraavat ehdot:~\cite{browne}

\begin{itemize}
	\item nollasummaisuus, eli kaikkien mahdollisten lopputulosten arvo pelaajille on yhteensä nolla. Tämä tarkoittaa, että yhden pelaajan saadessa paremman lopputuloksen muiden pelaajien lopputulos huononee.
	 \item täydellinen informaatio, eli tarkka pelitilanne on jatkuvasti kaikkien pelaajien tiedossa.
	 \item determinismi, eli peliin ei sisälly satunnaistekijöitä.
	 \item vuorottainen, eli pelaajat valitsevat ja suorittavat toimintansa vuoroittain.
	 \item diskreetti, eli peliä pelataan erillisissä vuoroissa reaaliaikaisuuden sijaan.
\end{itemize}

Esimerkkejä kombinatorisista peleistä ovat \textit{shakki} ja \textit{go}. Kombinatoriset pelit ovat tekoälylle erinomainen tutkimus- ja soveltamisala, sillä niissä on usein yksinkertaiset säännöt, joiden päälle rakentuu todella syvällistä ja monimutkaista strategiaa.~\cite{browne}

Kombinatorinen peli on helppo mallintaa pelipuuna. Pelipuun juurisolmu kuvaa pelin alkutilannetta, ja jokaisen solmun lapset pelitilanteita jotka solmun kuvaamasta pelitilanteesta voi saavuttaa. Pelin lehtisolmut ovat lopputilanteita, joissa pelin lopputulos määräytyy. Pelipuuna mallinnettua peliä on helppo käsitellä tavallisilla puuhakualgoritmeilla.~\cite{aima}

Perinteinen tapa toteuttaa pelitekoälyjä on \textit{minimax}-haku, jossa tekoäly olettaa kaikkien pelaajien pyrkivän maksimoimaan oman voittonsa toisten pelaajien pelatessa optimaalisesti~\cite{aima}. Tällainen tekoäly valitsisi siis mielummin varmasti tasapeliin johtavan siirron kuin sellaisen, jolla voi vastustajan valinnoista riippuen joko voittaa tai hävitä.

\textit{Minimax}-haun ongelmana on se, että vain puun juurisolmujen arvo pelaajille on tiedossa. Yksinkertaisissa peleissä, kuten 3x3-ristinollassa tämä ei ole vielä ongelma, mutta vaikkapa shakin pelipuu haarautuu aivan liian runsaasti jotta hakua voitaisiin jatkaa juurisolmuihin asti. Jotta seuraava siirto saataisiin valittua järkevässä ajassa on haku katkaistava ja käytettävä heuristiikkaa eli arviota katkaistun oksan todellisesta arvosta pelaajille.~\cite{aima}


\section{Monte Carlo -puuhaku}

Monte Carlo -puuhaussa (\textit{Monte Carlo Tree Search}, jatkossa \textbf{MCTS}) pelipuuta laajennetaan etenevästi, ja käsitellään leveyssuuntaisen etsinnän sijaan simuloimalla satunnaisilla siirroilla tilanteesta pelin etenemistä loppuun saakka. Menetelmän perusajatus on, että tarpeeksi suurella satunnaisotoksella voidaan arvioida kunkin siirron todellista hyötyä pelaajalle.~\cite{browne}

Menetelmässä puun solmut pitävät yllä tietoa odotusarvostaan, eli niiden kautta kulkeneiden simulaatiopelien tuloksista, ja siitä kuinka usein niitä on tutkittu. Pelipuuta laajennetaan valitsemalla solmu, jolla on vielä tutkimattomia lapsia. Solmun valintaa voi painottaa sen odotusarvon ja tutkimiskertojen mukaan, jolloin algoritmi käyttää vähemmän aikaa jo todennäköisesti huonojen tapausten tutkimiseen ja keskittyy uusien, kannattavien peliliikkeiden etsimiseen.~\cite{browne}

Valittua solmua laajentaessa sen lapsi tai lapset liitetään pelipuuhun, ja samassa yhteydessä simuloidaan satunnaisilla siirroilla peli uudesta lapsitilasta loppuun saakka. Pelin lopputilanteen arvo pelaajalle palautetaan pelipuussa takaisin juureen. Jokainen matkalla juureen oleva solmu käyttää simulaation tulosta oman odotusarvonsa päivittämiseen. ~\cite{browne}

MCTS valitsee siirron, kun haun laskenta-aika on kulutettu loppuun tai haku halutaan keskeyttää muusta syystä. Tämän jälkeen on vielä valittava, mikä siirto lopuksi valitaan. Eri malleja seuraavan siirron valintaan on useita. Voidaan valita esimerkiksi siirto, jolla on korkein odotusarvo, tai siirto jota on tutkittu eniten ja jonka odotusarvo on siten lähimpänä totuudenmukaista.~\cite{browne}


\section{Erityisiä hyötyjä}

MCTS on algoritmina erittäin helppo soveltaa monenlaisiin ongelmiin. Koska se ei tarvitse heuristiikkaa, tekoälyn ohjelmoijan ei tarvitse tietää pohjalla olevasta pelistä muuta kuin perussäännöt. Heuristiikan liittämisellä pelitilanteiden arviointiin voidaan kuitenkin saavuttaa parempia tuloksia myös MCTS:n kanssa.~\cite{browne}

MCTS ylläpitää arviota siirtojen kannattavuudesta reaaliaikaisesti. Haun voi siis keskeyttää koska tahansa ja saada silti kelvollisen ratkaisun. Pidemmän laskenta-ajan antaminen MCTS:lle parantaa kuitenkin tuloksen laatua, sillä algoritmilla on enemmän aikaa suorittaa simulaatioita ja siten etsiä parempia toimintamalleja. Ajankäyttö on myös tehokasta, sillä puuta voi laajentaa epätasaisesti painottaen lupaavia solmuja. ~\cite{browne}


% --- References ---
%
% bibtex is used to generate the bibliography. The babplain style
% will generate numeric references (e.g. [1]) appropriate for theoretical
% computer science. If you need alphanumeric references (e.g [Tur90]), use
%
% \bibliographystyle{babalpha-lf}
%
% instead.

\bibliographystyle{babplain-lf}
\bibliography{lahteet}


% --- Appendices ---

% uncomment the following

% \newpage
% \appendix
% 
% \section{Esimerkkiliite}

\end{document}

